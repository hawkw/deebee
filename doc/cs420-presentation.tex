\documentclass{beamer}
\usepackage{cleveref}
\usepackage{hyperref}
\usepackage{cite}
\usepackage{minted}

\mode<presentation>
{
  \usetheme{Bergen}
  \usecolortheme{crane}

  \setbeamercovered{transparent}
  % or whatever (possibly just delete it)
}


\usepackage[english]{babel}
% or whatever

\usepackage[latin1]{inputenc}
% or whatever

\usepackage{times}
\usepackage[T1]{fontenc}
% Or whatever. Note that the encoding and the font should match. If T1
% does not look nice, try deleting the line with the fontenc.


\title{Implementation of a Relational Database Query-Processing System}

\author{Hawk Weisman}

\institute[Allegheny College] % (optional, but mostly needed)
{Department of Computer Science\\Allegheny College
}

\date{December 8th, 2014}

\subject{Compilers, Databases}
% This is only inserted into the PDF information catalog. Can be left
% out. 




% Delete this, if you do not want the table of contents to pop up at
% the beginning of each subsection:
\AtBeginSubsection[]
{
  \begin{frame}<beamer>{Outline}
    \tableofcontents[currentsection,currentsubsection]
  \end{frame}
}


% If you wish to uncover everything in a step-wise fashion, uncomment
% the following command: 

%\beamerdefaultoverlayspecification{<+->}


\begin{document}

\begin{frame}
  \titlepage
\end{frame}

\begin{frame}{Outline}
  \tableofcontents
  % You might wish to add the option [pausesections]
\end{frame}


% Structuring a talk is a difficult task and the following structure
% may not be suitable. Here are some rules that apply for this
% solution: 

% - Exactly two or three sections (other than the summary).
% - At *most* three subsections per section.
% - Talk about 30s to 2min per frame. So there should be between about
%   15 and 30 frames, all told.

% - A conference audience is likely to know very little of what you
%   are going to talk about. So *simplify*!
% - In a 20min talk, getting the main ideas across is hard
%   enough. Leave out details, even if it means being less precise than
%   you think necessary.
% - If you omit details that are vital to the proof/implementation,
%   just say so once. Everybody will be happy with that.

\section{Background}

\subsection{Relational Databases}

\begin{frame}{What is a relational database?}
  % - A title should summarize the slide in an understandable fashion
  %   for anyone how does not follow everything on the slide itself.

  \begin{itemize}
  \item ``The primary data model for commercial data-processing applications.''~\cite{silberschatz2010database} \pause
  \item A {\it relation} is a table of values~\cite{silberschatz2010database,garcia2000database} \pause
  \item A relation consists of: \pause
  \begin{itemize}
    \item a set of rows, or {\it tuples}~\cite{silberschatz2010database,garcia2000database} \pause
     \item a set of columns, or {\it attributes}~\cite{silberschatz2010database,garcia2000database} \pause
  \end{itemize} 
  \item A database can consist of multiple relations~\cite{silberschatz2010database,garcia2000database}
  \item So how does this relate to compilers?
  \end{itemize}
\end{frame}

\subsection{Query Languages}
\begin{frame}{What is a query langauge?}
  \begin{itemize}
  \item Users and client software interact with databases through {\it query languages}~\cite{silberschatz2010database,garcia2000database} \pause
  \item These are {\it domain-specific languages} for accessing and modifying the database \pause
  \item Query languages are {\it declarative} rather than {\it imperative} programming languages \pause
  \item Just like other programming languages, query languages must be parsed, analyzed, and compiled or interpreted.
  \end{itemize}
\end{frame}

\begin{frame}{SQL}
\begin{itemize}
  \item SQL is the {\it Structured Query Language}. \pause
  \item It is the query language used by most modern RDBMSs \pause
  \item SQL consists of two components: \pause
    \begin{itemize}
      \item {\it Data definition language} (DDL): defines the structure of the database 
      \begin{itemize}
        \item creating and deleting tables
        \item adding relationships betweent tables
        \item et cetera
        \end{itemize}\pause
      \item {\it Data manipulation language} (DML): accesses and modifies data stored in the database
      \begin{itemize}
        \item selecting rows
        \item adding, deleting, and modifying rows
        \item et cetera
      \end{itemize}
    \end{itemize} \pause
    \item SQL = DDL + DML
\end{itemize}
\end{frame}


\subsection{Handling Queries}

\begin{frame}{Query Parsing}
\end{frame}

\begin{frame}{Query Processing}
\end{frame}



\section{DeeBee}

\subsection{Architecture}



\subsection{Implementation}



\section*{Summary}

\begin{frame}{Summary}

\end{frame}



% All of the following is optional and typically not needed. 
\appendix
\section<presentation>*{\appendixname}
\subsection<presentation>*{References}

\begin{frame}[allowframebreaks]
        \frametitle{References}
\bibliography{assets/final}{}
\bibliographystyle{plain}
\end{frame}

\end{document}


